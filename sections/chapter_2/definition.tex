
\subsection{Définition de l'intelligence}

L'intelligence est souvent vue comme étant propre à l'être humain, on emploie d'ailleurs le terme <<~bête~>> pour désigner à la fois les animaux et les personnes que l'on considère comme ayant peu d'intelligence.

Pourtant, on pourrait la définir comme étant la capacité à s'adapter à des situations nouvelles, ce qui permet de prendre en compte de nombreux comportements que l'on trouve chez les animaux, et même plus globalement chez les êtres vivants \cite{ia_pour_dev}.

En exemple de cette capacité d'adaptation, on peut citer le cas de Hans le malin, un cheval qui <<~savait~>> compter et donnait le résultat d'un calcul par des coups de sabot sur le sol.
En réalité, le cheval n'était en capacité ni de comprendre la question ni de calculer la réponse, mais il était capable de détecter les micro-expressions du public ou de son maître pour savoir quand s'arrêter de frapper.

Une autre forme d'intelligence intéressante est l'intelligence collective, on la retrouve chez les humains, dans des mouvements de foule notamment, mais aussi chez beaucoup d'animaux et insectes.
Les fourmis sont un autre très bon exemple avec la présence de rôles~:~reine, ouvrières, nourrices, gardiennes, combattantes, etc.

Il existe encore beaucoup d'autres formes d'intelligences et toutes celles-ci sont très intéressantes à analyser et à imiter à l'aide de l'intelligence artificielle.
Nous verrons plus en détail par la suite les mises en application possibles.

\subsection{Définition de l'intelligence artificielle}

Dans leur livre <<~Artificial Intelligence:~A Modern Approach~>> \cite{modern_approach}, Stuart J. \textsc{Russell} et Peter \textsc{Norvig}, deux acteurs importants de la recherche en intelligence artificielle, la définissent sous quatre aspects différents.

Ces aspects opposent le fait de penser au fait d'agir et le fait de le faire humainement ou rationnellement.
Ces quatre approches ont toutes été suivies par différentes personnes en employant des méthodes différentes dans leurs recherches.

\FloatBarrier
\begin{figure}[h!]
    \begin{center}
        \begin{tabular}{ |m{0.4\textwidth}|m{0.4\textwidth}| }
            \hline
            \textbf{Penser humainement} & \textbf{Penser rationnellement} \\
            &\\
            Système développé dans l'idée d'imiter la manière de penser humaine & Système développé dans l'idée de penser de manière rationnelle \\
            \hline
            \textbf{Agir humainement} & \textbf{Agir rationnellement} \\
            &\\
            Système développé dans l'idée d'agir comme un humain le ferait face à une même situation & Système développé dans l'idée d'agir de la manière la plus rationnelle possible \\
            \hline
        \end{tabular}
    \end{center}
    \caption{Quatre approches de l'intelligence artificielle}
    \label{figure:ai_approaches}
\end{figure}
\FloatBarrier

\subsubsection{Penser humainement, la science cognitive}

Avant de pouvoir faire en sorte qu'un programme pense comme un être humain, il faut déjà être en mesure de déterminer comment un humain pense.

Il y a trois manières pour cela~:~
\\
\begin{enumerate}
    \item L'introspection~:~essayer d'analyser nos pensées et notre manière de penser.
    \item L'expérimentation psychologique~:~observer une personne en action.
    \item L'imagerie cérébrale~:~observer les stimuli d'un cerveau en action.
\end{enumerate}
~\\

\subsubsection{Penser rationnellement, les lois de la pensée}

Le philosophe \textsc{Aristote} fut le premier à réfléchir à ce qu'est <<~penser rationnellement~>>.
Il est à l'origine du syllogisme qu'il définit comme <<~un discours dans lequel, certaines choses étant posées, quelque chose d'autre que ces données en résulte nécessairement par le seul fait de ces données~>> \cite{aristote}.

Le syllogisme repose sur deux prémisses, une majeure et une mineure, desquelles on peut tirer une conclusion, par exemple~:~

\begin{quote}
    Majeure~:~\textsc{Socrate} est un homme.
    
    Mineure~:~Les hommes sont mortels.
    
    Conclusion~:~\textsc{Socrate} est mortel.
\end{quote}

\subsubsection{Agir humainement, le test de \textsc{Turing}}

En 1950, dans son livre <<~Computer and thought~>> \cite{computer_and_thought}, Alan \textsc{Turing} décrit un test d'intelligence artificielle consistant à mettre un humain en confrontation, à l’aveugle, avec un ordinateur et un autre humain.
Avec cet ouvrage il souhaitait remplacer la question <<~les machines peuvent-elles penser~?~>> par <<~un ordinateur digital peut-il tenir la place d'un être humain dans le jeu de l'imitation~?~>>, la première n'ayant que trop peu de sens pour mériter une discussion \cite{test_turing}.

Le test se présente comme tel~:~

\begin{quote}
    Un être humain noté $C$ est confronté à l'aveugle à un ordinateur et un autre être humain, ces derniers sont notés $A$ et $B$ sans que $C$ ne sache ce qu'ils sont.
    $C$ peut poser des questions à $A$ et $B$, ceux-ci répondent via un message écrit à l'ordinateur afin que le ton de la voix n'aide pas $C$.
    Si $C$ n'est pas en mesure de déterminer lequel est humain ou non, l'ordinateur passe le test.
    Les résultats de ce test ne dépendent pas de la capacité de l'ordinateur à donner de bonnes réponses, mais plutôt de la capacité à imiter les réponses d'un humain.
\end{quote}

Pour le moment, aucun programme n'est parvenu à passer ce test, il reste donc encore pertinent.
Cependant, les chercheurs du domaine n'ont pas beaucoup cherché à passer ce test, préférant se concentrer sur l'étude des principes régissant l'intelligence plutôt que d'essayer d'imiter cette dernière.
Effectivement, d'autres domaines scientifiques ont montré de gros progrès en appliquant cette méthode.
Par exemple, c'est lorsque les frères \textsc{Wright} ont cessé d'essayer d'imiter les oiseaux et se sont concentrés sur le principe sous-jacent d'aérodynamique qu'ils sont parvenus à créer une machine volante.

\subsubsection{Agir rationnellement, les agents rationnels}

Un agent est, par définition, simplement quelque chose qui agit.
Un agent rationnel se doit cependant de prendre les décisions qui sont supposées lui permettre de maximiser sa mesure de performance.
Il doit également être en mesure d'agir de manière autonome, de percevoir son environnement et de persister sur une durée de temps prolongée sans nécessiter aucune aide extérieure.
Contrairement à la catégorie <<~penser rationnellement~>>, ici l'agent n'est pas obligé de penser avant d'agir, il peut faire preuve de <<~réflexes~>> si cette action est la plus rationnelle.

\subsubsection{Conclusion}

Il existe donc des chercheurs qui pensent qu'une intelligence artificielle se doit de penser ou d'agir comme un être humain et d'autres que celle-ci doit penser ou agir rationnellement.
Il n'est pas possible de dire lesquels ont raison, cela relève du débat philosophique et éthique.

L'encyclopédie Larousse la définit comme <<~l'ensemble des théories et des techniques mises en œuvre en vue de réaliser des machines capables de simuler l'intelligence~>> \cite{def_larousse}.
Cette définition est clairement basée sur le concept d'agir de manière humaine comme Alan \textsc{Turing} la définissait.

L'intelligence en elle-même est un concept difficile à définir précisément, car celle-ci peut prendre de nombreuses formes.
Il existe des tests qui tentent de mesurer l'intelligence, ce sont les tests de Q.I.
Mais ceux-ci sont sujets à de fortes polémiques, car avec un entraînement spécifique une personne peut fortement améliorer ses scores à ce genre de tests.
Il en va de même avec le test de \textsc{Turing} pour l'intelligence artificielle.