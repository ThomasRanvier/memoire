
J'ai effectué ma huitième et dernière mission en entreprise, dans le cadre de ma formation d'ingénieur en informatique à l'école 3iL, de février à septembre 2019.
Lors de mes trois années de formation, mon apprentissage s'est déroulé au sein du service informatique de la compagnie \textsc{Disa}.
Celle-ci est une imprimerie de labeur basée à Limoges, spécialisée dans l'impression de publicité sur lieu de vente.

Dans ce mémoire, nous allons nous intéresser à la problématique <<~\sujet~>>.
L'intérêt de cette problématique est de corréler un sujet qui me passionne et dans lequel je vais continuer mes études, l'intelligence artificielle, au sujet des ERP, sachant que j'ai travaillé sur le développement d'un ERP dans le cadre de mon apprentissage.

Ce mémoire est divisé en deux principales parties.

La première consiste en un rapport d'activité de mon apprentissage.
Le but est d'y décrire le travail que j'ai effectué durant ces trois années.
Je vais tout d'abord présenter l'entreprise \textsc{Disa}, puis le projet \textsc{Sigma}, projet de développement d'un ERP sur lequel j'ai travaillé durant la totalité de mon apprentissage.

La seconde partie est consacrée à mon travail de recherche sur la problématique citée ci-dessus.
Dans un premier temps, nous verrons un historique, ainsi que l'état de l'art de l'intelligence artificielle.
Ensuite, nous allons aborder différentes applications possibles de l'intelligence artificielle au sein des ERP.
Nous comparerons plusieurs algorithmes pour chaque application afin de déterminer les avantages et inconvénients de chacun d'entre eux sans pour autant aller trop loin d'un point de vue technique. 
