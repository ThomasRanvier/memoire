
J'ai eu la chance d'intégrer la compagnie \textsc{Disa} au départ du projet \textsc{Sigma}, et ainsi, de pouvoir travailler sur toutes les étapes de son développement.
Allant de la conception, jusqu'à assister à sa mise en production et à son utilisation par les employés, désormais quotidienne.
Mon travail sur ce projet fut très varié, ce qui a rendu mon apprentissage fortement intéressant.
Au début de mon apprentissage, M. \textsc{Palier} s'occupait de toute la partie récolte et analyse des besoins, tandis que je me concentrais sur le développement et la prise en main des outils de développement.
Puis, avec le temps, la confiance de M. \textsc{Palier} à mon égard s'est instaurée et il me laissa de plus en plus d'autonomie dans mon travail, en me laissant m'occuper de récolter et analyser les besoins des employés.
Cette autonomie m'a aidée à améliorer mes compétences relationnelles, ma confiance personnelle, ainsi que mes capacités d'adaptation à des situations nouvelles.

Mon travail m'a permis de découvrir le fonctionnement d'une PME industrielle et de nombreux métiers au sein de celle-ci.
J'ai énormément appris en analysant le travail des employés afin de pouvoir concevoir des outils qui leur soient adaptés.

Nous avons vu dans ce mémoire un certain nombre de concepts connus de l'intelligence artificielle pouvant s'appliquer aux ERP.
Nous avons notamment abordé des méthodes de classification et de partitionnement automatique qui peuvent être utilisées pour classer automatiquement des documents que l'on importe au sein d'une GED.
L'avantage d'utiliser une méthode de classification intelligente au sein d'une GED serait de grandement simplifier les imports de fichiers.
Différentes méthodes ont été présentées et analysées afin de déterminer les avantages et inconvénients de chacune d'entre elles.
Nous avons évoqué des outils permettant de détecter des informations importantes au sein d'un document afin de pouvoir les extraire et les enregistrer directement en base de données, sans nécessiter de saisie manuelle par un utilisateur.
Nous avons vu plusieurs méthodes de synthèse de texte et décrit le fonctionnement de la synthèse par abstraction.
Pour finir, nous avons brièvement parlé d'applications possibles de l'intelligence artificielle concernant la gestion des stocks ainsi que les assistants intelligents.

Les domaines d'application de l'intelligence artificielle sont vastes et nous n'en avons abordé qu'une infime partie avec cette problématique concernant les ERP.

Je vais continuer mes études dans le domaine de l'intelligence artificielle, plus précisément en Master 2 en intelligence artificielle à l'université \textsc{Claude Bernard} de Lyon, la problématique du sujet de recherche m'a donc particulièrement intéressée.
La rédaction de ce mémoire m'a aussi permis d'élargir mes connaissances de l'historique et de l'état de l'art de ce domaine, ce qui me sera fortement bénéfique.
