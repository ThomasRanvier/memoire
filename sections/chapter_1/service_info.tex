\subsection{Le service informatique}

Le service informatique de \textsc{Disa} est d'effectif très restreint et sa composition a beaucoup évolué durant mes trois années d'apprentissage.

Les deux premières années, M. Florent \textsc{Palier} en était le responsable. 
Il fut responsable du service informatique de \textsc{Disa} de 2001 jusque fin 2018 avant de devenir un employé de \textsc{DisaTech} après son rachat par AkzoNobel.
En préparation de son départ, M. Nicolas \textsc{Stecleboute} fut recruté en septembre 2018 et formé par Florent \textsc{Palier} pour devenir le nouveau responsable du service.

À mon arrivée au sein de \textsc{Disa}, en septembre 2016, le service comprenait M. \textsc{Palier}, ainsi qu'un autre apprenti, Alexandre \textsc{Meunier}.
Celui-ci était étudiant en dernière année de master à l'école 3iL Academy.
Durant les mois de mai et juin 2017, le service comprit aussi un stagiaire étudiant en informatique à l'IUT de Limoges, Jeremy \textsc{Coudert}.
Florent \textsc{Palier} avait souvent recours à des apprentis et stagiaires lorsque la charge de travail du service était plus haute.

En septembre 2017, Alexandre \textsc{Meunier} termina son apprentissage au sein de \textsc{Disa}, nous n'étions ensuite plus que deux avec M. Florent \textsc{Palier}.

Depuis mon retour de ma mobilité en Suède, en janvier 2019, nous sommes deux au service informatique avec Nicolas \textsc{Stecleboute}.
Florent \textsc{Palier} reste mon tuteur en entreprise, même s'il ne fait techniquement plus partie de \textsc{Disa}.
Il travaille désormais de son côté pour \textsc{DisaTech}.
Cependant il participe toujours au projet \textsc{Sigma}, le projet principal du service sur ces trois dernières années, qui consiste en la création d'un nouvel ERP\footnotemark.
\footnotetext{ERP~:~Enterprise Ressource Planning, détaillé dans la partie <<~Le projet \textsc{Sigma}~>>}
Il est en train de préparer une version de \textsc{Sigma} pleinement adapté à sa compagnie afin que cette version ne dépende plus de celle de \textsc{Disa}.

Les membres du service informatique s'assurent du bon fonctionnement du système d'information de la compagnie.
Nous sommes responsables du matériel informatique, des télécommunications, du réseau, des serveurs, ainsi que de tous les logiciels et les applications utilisées par les employés de la compagnie.
À ce travail quotidien s'ajoute le développement de nouveaux outils informatiques destinés à améliorer les conditions de travail de nos collaborateurs.

\subsubsection{L’infrastructure matérielle}

Le parc informatique de \textsc{Disa} est principalement constitué de PC fonctionnant sous Windows 10 ou des versions plus anciennes.
Les membres du pôle d'infographie sont les seuls n'utilisant pas Windows.
Ils utilisent des Mac et ont également leurs propres serveurs de stockage réservés à leur service.
Chaque employé en ayant le besoin possède un téléphone relié au réseau interne de la compagnie.

En 2014, \textsc{Disa} a fait l'acquisition d'un mini data center, un PowerEdge VRTX, système permettant notamment la virtualisation de serveurs et le stockage de données.
Ainsi, tous les serveurs de la compagnie sont centralisés sous cette machine, il regroupe~:~
\\
\begin{itemize}
    \item[\tiny$\bullet$] Quatre disques de 1To servant pour le stockage des données des serveurs partagés.
    \item[\tiny$\bullet$] Un serveur Windows Server 2010 qui héberge Sage, le logiciel servant à la comptabilité.
    \item[\tiny$\bullet$] Un serveur sous Linux permettant la réalisation de tâches planifiées.
    \item[\tiny$\bullet$] Un serveur web hébergeant l'intranet.
    \item[\tiny$\bullet$] Un serveur de test pour le développement du nouvel ERP \textsc{Sigma}.
    \item[\tiny$\bullet$] Un serveur de production pour le déploiement et l'utilisation de \textsc{Sigma}.
\end{itemize}
~\\

Toutes les données sur les disques du VRTX sont protégées par un système RAID 5, ce système répartit les données sur plusieurs disques de manière à pouvoir récupérer toutes les informations même si un disque venait à tomber en panne.

Pour le moment, la compagnie \textsc{DisaTech} est encore reliée au réseau de \textsc{Disa} par l'intermédiaire d'un canon Wi-Fi, car l'accès au serveur de stockage de \textsc{Disa} leur est encore nécessaire.
Cependant dès lors que le réseau et la gestion des serveurs du côté de \textsc{DisaTech} seront entièrement opérationnels et indépendants, cette connexion n'aura plus lieu d'être et sera supprimée.

\subsubsection{L'environnement logiciel de \textsc{Disa}}

L'environnement logiciel de la compagnie a évolué durant mon apprentissage et est encore en cours de mutation.
Nous sommes en train de développer un nouvel ERP pour l'entreprise, nommé \textsc{Sigma}, l'objectif de ce projet est de fournir un outil unique aux employés afin de faciliter les liens entre chaque étape de la fabrication.

Actuellement, les employés utilisent principalement \textsc{Sigma} pour leur travail.
En parallèle, ils utilisent encore ASAP (l'ancien ERP qui vise à être remplacé) pour effectuer des tâches qui n'ont pas encore été implémentées sous \textsc{Sigma}.
L'intranet de l'entreprise, \textsc{Disanet}, est encore utilisé par certains employés au même titre qu'ASAP, afin d'effectuer des tâches pour le moment non réalisables sous \textsc{Sigma}.
Pour la comptabilité, le logiciel \textsc{Sage} est utilisé, celui-ci est un logiciel propriétaire, nous ne pouvons donc pas le modifier.
\\
\begin{itemize}
    \item[\tiny$\bullet$] \textbf{ASAP~:}~C'est l'ERP utilisé à \textsc{Disa} depuis l'année 2002, jusqu'au passage sous \textsc{Sigma}.
    Permettant la gestion complète de tous les flux internes de \textsc{Disa}, il est devenu rapidement indispensable pour gérer la production dans de bonnes conditions depuis son installation.
    
    Cependant, avec le temps, des besoins non gérés par ce dernier sont apparus et c'est pourquoi en 2006 un intranet fut créé.
    
    \item[\tiny$\bullet$] \textbf{\textsc{Disanet}~:}~C'est l'intranet de l'entreprise, il fut créé en 2006 par un stagiaire de l'école 3iL.
    Dans un premier temps, \textsc{Disanet} fut vu comme une opportunité permettant de combler les manques d’ASAP.
    
    C’est pour cela que, depuis sa création, une centaine d'applications ont été développées sur ce support~:~il est donc devenu un complément indispensable d’ASAP.
    Désormais, le nombre d'applications sur \textsc{Disanet} est tel que les employés sont constamment amenés à passer de \textsc{Disanet} à ASAP et inversement, ce qui ne facilite pas leur travail.
    
    C’est pour cette raison que M. \textsc{Palier} a décidé de commencer le projet \textsc{Sigma} en 2016.
    Ce dernier vise à centraliser toutes les applications métiers, actuellement éparpillées entre l’ERP et l’intranet, au sein d’un unique outil.
    M. \textsc{Palier} souhaite donc que l’intranet retrouve un rôle de communication plutôt qu’un rôle applicatif et que nous cessions d'utiliser ASAP.
    
    \item[\tiny$\bullet$] \textbf{\textsc{Sigma}~:}~Depuis la fin de l'année 2018, les employés ont commencé à utiliser ce nouvel ERP quotidiennement.
    Le passage à l'utilisation de celui-ci s'est fait progressivement afin d'avoir le temps de former les employés à l'utilisation de l'outil et de ne pas générer une affluence de problèmes à résoudre trop importante.
    Cette migration a continué durant quelques mois.
    Désormais, les employés ayant encore besoin d'utiliser ASAP et \textsc{Disanet} sont rares.
    Il reste encore des applications à développer et intégrer sur \textsc{Sigma} et d'autres à corriger et/ou améliorer (ce travail est en cours).
    
    \item[\tiny$\bullet$] \textbf{SFAO~:}~Avant l'utilisation de \textsc{Sigma}, les ouvriers utilisaient la SFAO~:~Suivi de Fabrication Assisté par Ordinateur.
    Cet outil permettait de superviser la production et de garder un historique de la fabrication.
    Concrètement, il permettait de savoir quel salarié avait travaillé sur chaque tache de fabrication.
    Il permettait aussi de récupérer des informations permettant de générer des statistiques sur le processus de production et de gérer plus facilement les soucis rencontrés par les ouvriers.
    
    Cet outil est désormais entièrement intégré au sein de \textsc{Sigma}, l'ancien système n'est plus utilisé du tout.
\end{itemize}
~\\

\subsubsection{Mon rôle au sein du service}

Durant mes trois années d'apprentissage, la tâche principale qui m'a été confiée est la participation à la conception, au développement et à la mise en place du projet \textsc{Sigma}.
C'est un projet de grande ampleur, et travailler sur les différentes phases de celui-ci m'a permis de découvrir de nombreux aspects différents du travail de développeur et de chef de projet.

Mon quotidien fut globalement découpé entre l'analyse des besoins des futurs utilisateurs de \textsc{Sigma} et le développement d'applications adaptées pour leur permettre de travailler le plus simplement possible.
Une fois la mise à disposition de \textsc{Sigma} aux employés, s'en est suivi une période focalisée sur la résolution de bugs et l'amélioration des applications afin de les faire mieux correspondre aux besoins de chacun.

En plus de ce travail quotidien, je porte aussi assistance aux employés lorsqu'ils rencontrent des problèmes informatiques.

L'effectif très réduit du service oblige souvent à être confronté à des problèmes jamais rencontrés auparavant et sollicite la mise en application de compétences variées~:~cela favorise la polyvalence.